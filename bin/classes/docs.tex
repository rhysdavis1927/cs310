\documentclass[11pt]{report}
\def\bl{\mbox{}\newline\mbox{}\newline{}}
\usepackage{ifthen}
\newcommand{\hide}[2]{
\ifthenelse{\equal{#1}{inherited}}%
{}%
{}%
}
\newcommand{\entityintro}[3]{%
  \hbox to \hsize{%
    \vbox{%
      \hbox to .2in{}%
    }%
    {\bf #1}%
    \dotfill\pageref{#2}%
  }
  \makebox[\hsize]{%
    \parbox{.4in}{}%
    \parbox[l]{5in}{%
      \vspace{1mm}\it%
      #3%
      \vspace{1mm}%
    }%
  }%
}
\newcommand{\isep}[0]{%
\setlength{\itemsep}{-.4ex}
}
\newcommand{\sld}[0]{%
\setlength{\topsep}{0em}
\setlength{\partopsep}{0em}
\setlength{\parskip}{0em}
\setlength{\parsep}{-1em}
}
\newcommand{\headref}[3]{%
\ifthenelse{#1 = 1}{%
\addcontentsline{toc}{section}{\hspace{\qquad}\protect\numberline{}{#3}}%
}{}%
\ifthenelse{#1 = 2}{%
\addcontentsline{toc}{subsection}{\hspace{\qquad}\protect\numerline{}{#3}}%
}{}%
\ifthenelse{#1 = 3}{%
\addcontentsline{toc}{subsubsection}{\hspace{\qquad}\protect\numerline{}{#3}}%
}{}%
\label{#3}%
\makebox[\textwidth][l]{#2 #3}%
}%
\newcommand{\membername}[1]{{\it #1}\linebreak}
\newcommand{\divideents}[1]{\vskip -1em\indent\rule{2in}{.5mm}}
\newcommand{\refdefined}[1]{
\expandafter\ifx\csname r@#1\endcsname\relax
\relax\else
{$($ in \ref{#1}, page \pageref{#1}$)$}
\fi}
\newcommand{\startsection}[4]{
\gdef\classname{#2}
\subsection{\label{#3}{\bf {\sc #1} #2}}{
\rule[1em]{\hsize}{4pt}\vskip -1em
\vskip .1in 
#4
}%
}
\newcommand{\startsubsubsection}[2]{
\subsubsection{\sc #1}{%
\rule[1em]{\hsize}{2pt}%
#2}
}
\usepackage{color}
\date{\today}
\pagestyle{myheadings}
\addtocontents{toc}{\protect\def\protect\packagename{}}
\addtocontents{toc}{\protect\def\protect\classname{}}
\markboth{\protect\packagename -- \protect\classname}{\protect\packagename -- \protect\classname}
\oddsidemargin 0in
\evensidemargin 0in
% \topmargin -.8in
\chardef\bslash=`\\
\textheight 9.4in
\textwidth 6.5in
\begin{document}
\sloppy
\raggedright
\tableofcontents
\gdef\packagename{}
\gdef\classname{}
\newpage
\def\packagename{logic.proof.builder.proof}
\chapter{\bf Package logic.proof.builder.proof}{
\vskip -.25in
\hbox to \hsize{\it Package Contents\hfil Page}
\rule{\hsize}{.7mm}
\vskip .13in
\hbox{\bf Classes}
\entityintro{Proof}{l0}{Stores all data necessary to construct a proof.}
\entityintro{ProofStep}{l1}{...no description...}
\entityintro{RulesOfInference}{l2}{Contains methods for the rules of inference of first-order logic}
\vskip .1in
\rule{\hsize}{.7mm}
\vskip .1in
\newpage
\section{Classes}{
\startsection{Class}{Proof}{l0}{%
{\small Stores all data necessary to construct a proof. Simple Methods are provided to manipulate
 the proof such as adding or deleting lines.}
\vskip .1in 
\startsubsubsection{Declaration}{
\fbox{\vbox{
\hbox{\vbox{\small public 
class 
Proof}}
\noindent\hbox{\vbox{{\bf extends} java.lang.Object}}
}}}
\startsubsubsection{Fields}{
\begin{itemize}
\item{
public List predicates\begin{itemize}\item{\vskip -.9ex A list of all the named predicates in the proof. Used to populate the
 predicate list.}\end{itemize}
}
\end{itemize}
}
\startsubsubsection{Constructors}{
\vskip -2em
\begin{itemize}
\item{\vskip -1.9ex 
\membername{Proof}
{\tt public {\bf Proof}(  )
\label{l3}\label{l4}}%end signature
\begin{itemize}
\sld
\item{
\sld
{\bf Usage}
  \begin{itemize}\isep
   \item{
Default constructor. Constructs an empty proof
}%end item
  \end{itemize}
}
\end{itemize}
}%end item
\end{itemize}
}
\startsubsubsection{Methods}{
\vskip -2em
\begin{itemize}
\item{\vskip -1.9ex 
\membername{addStepAsEndOfSubproof}
{\tt public ProofStep {\bf addStepAsEndOfSubproof}( {\tt logic.proof.builder.parser.SimpleNode } {\bf node},
{\tt java.lang.String } {\bf formula} )
\label{l5}\label{l6}}%end signature
\begin{itemize}
\sld
\item{
\sld
{\bf Usage}
  \begin{itemize}\isep
   \item{
Adds a new proofstep that is the last line of a subproof
}%end item
  \end{itemize}
}
\item{
\sld
{\bf Parameters}
\sld\isep
  \begin{itemize}
\sld\isep
   \item{
\sld
{\tt node} - Root node of the sentence of the proofstep}
   \item{
\sld
{\tt formula} - String representation of the sentence}
  \end{itemize}
}%end item
\item{{\bf Returns} - 
Returns the proofstep that has been added 
}%end item
\end{itemize}
}%end item
\divideents{addStepAsNewLine}
\item{\vskip -1.9ex 
\membername{addStepAsNewLine}
{\tt public ProofStep {\bf addStepAsNewLine}( {\tt logic.proof.builder.parser.SimpleNode } {\bf node},
{\tt java.lang.String } {\bf formula} )
\label{l7}\label{l8}}%end signature
\begin{itemize}
\sld
\item{
\sld
{\bf Usage}
  \begin{itemize}\isep
   \item{
The default method to add a new proofstep to the proof
}%end item
  \end{itemize}
}
\item{
\sld
{\bf Parameters}
\sld\isep
  \begin{itemize}
\sld\isep
   \item{
\sld
{\tt node} - Root node of the sentence of the proofstep}
   \item{
\sld
{\tt formula} - String representation of the sentence}
  \end{itemize}
}%end item
\item{{\bf Returns} - 
Returns the proofstep that has been added 
}%end item
\end{itemize}
}%end item
\divideents{addStepAsStartOfSubproof}
\item{\vskip -1.9ex 
\membername{addStepAsStartOfSubproof}
{\tt public ProofStep {\bf addStepAsStartOfSubproof}( {\tt logic.proof.builder.parser.SimpleNode } {\bf node},
{\tt java.lang.String } {\bf formula} )
\label{l9}\label{l10}}%end signature
\begin{itemize}
\sld
\item{
\sld
{\bf Usage}
  \begin{itemize}\isep
   \item{
Adds a new proofstep that is the start of a subproof
}%end item
  \end{itemize}
}
\item{
\sld
{\bf Parameters}
\sld\isep
  \begin{itemize}
\sld\isep
   \item{
\sld
{\tt node} - Root node of the sentence of the proofstep}
   \item{
\sld
{\tt formula} - String representation of the sentence}
  \end{itemize}
}%end item
\item{{\bf Returns} - 
Returns the proofstep that has been added 
}%end item
\end{itemize}
}%end item
\divideents{addVar}
\item{\vskip -1.9ex 
\membername{addVar}
{\tt public ProofStep {\bf addVar}( {\tt java.lang.String } {\bf var} )
\label{l11}\label{l12}}%end signature
\begin{itemize}
\sld
\item{
\sld
{\bf Usage}
  \begin{itemize}\isep
   \item{
Add a new proofstep which introduces a boxed variable
}%end item
  \end{itemize}
}
\item{
\sld
{\bf Parameters}
\sld\isep
  \begin{itemize}
\sld\isep
   \item{
\sld
{\tt var} - The name of the variable being introduced}
  \end{itemize}
}%end item
\item{{\bf Returns} - 
Returns the proofstep that has been added 
}%end item
\end{itemize}
}%end item
\divideents{addVar}
\item{\vskip -1.9ex 
\membername{addVar}
{\tt public ProofStep {\bf addVar}( {\tt java.lang.String } {\bf introducedVariable},
{\tt logic.proof.builder.parser.SimpleNode } {\bf rootNode},
{\tt java.lang.String } {\bf formula} )
\label{l13}\label{l14}}%end signature
\begin{itemize}
\sld
\item{
\sld
{\bf Usage}
  \begin{itemize}\isep
   \item{
Add a new proofstep which introduces a boxed variable alongside an
 assumption
}%end item
  \end{itemize}
}
\item{
\sld
{\bf Parameters}
\sld\isep
  \begin{itemize}
\sld\isep
   \item{
\sld
{\tt introducedVariable} - The name of the variable being introduced}
   \item{
\sld
{\tt node} - Root node of the sentence}
   \item{
\sld
{\tt formula} - String representation of the sentence}
  \end{itemize}
}%end item
\item{{\bf Returns} - 
Returns the proofstep that has been added 
}%end item
\end{itemize}
}%end item
\divideents{getCurrentLevel}
\item{\vskip -1.9ex 
\membername{getCurrentLevel}
{\tt public int {\bf getCurrentLevel}(  )
\label{l15}\label{l16}}%end signature
\begin{itemize}
\sld
\item{
\sld
{\bf Usage}
  \begin{itemize}\isep
   \item{
Returns the number of subproofs currently open
}%end item
  \end{itemize}
}
\item{{\bf Returns} - 
the number of subproofs currently open 
}%end item
\end{itemize}
}%end item
\divideents{getLines}
\item{\vskip -1.9ex 
\membername{getLines}
{\tt public ArrayList {\bf getLines}(  )
\label{l17}\label{l18}}%end signature
\begin{itemize}
\sld
\item{
\sld
{\bf Usage}
  \begin{itemize}\isep
   \item{
Returns the ordered list of proofsteps
}%end item
  \end{itemize}
}
\item{{\bf Returns} - 
the ordered list of proofsteps 
}%end item
\end{itemize}
}%end item
\divideents{removeStep}
\item{\vskip -1.9ex 
\membername{removeStep}
{\tt public void {\bf removeStep}(  )
\label{l19}\label{l20}}%end signature
\begin{itemize}
\sld
\item{
\sld
{\bf Usage}
  \begin{itemize}\isep
   \item{
Removes the most recent line from the proof
}%end item
  \end{itemize}
}
\end{itemize}
}%end item
\end{itemize}
}
}
\startsection{Class}{ProofStep}{l1}{%
\startsubsubsection{Declaration}{
\fbox{\vbox{
\hbox{\vbox{\small public 
class 
ProofStep}}
\noindent\hbox{\vbox{{\bf extends} java.lang.Object}}
}}}
\startsubsubsection{Fields}{
\begin{itemize}
\item{
public ProofStep parent\begin{itemize}\item{\vskip -.9ex }\end{itemize}
}
\item{
public List subproofs\begin{itemize}\item{\vskip -.9ex }\end{itemize}
}
\item{
public ProofStep next\begin{itemize}\item{\vskip -.9ex }\end{itemize}
}
\item{
public SimpleNode node\begin{itemize}\item{\vskip -.9ex }\end{itemize}
}
\item{
public Integer lineNumber\begin{itemize}\item{\vskip -.9ex }\end{itemize}
}
\item{
public int level\begin{itemize}\item{\vskip -.9ex }\end{itemize}
}
\item{
public String formula\begin{itemize}\item{\vskip -.9ex }\end{itemize}
}
\item{
public String justification\begin{itemize}\item{\vskip -.9ex }\end{itemize}
}
\item{
public boolean endOfSubproof\begin{itemize}\item{\vskip -.9ex }\end{itemize}
}
\item{
public HashMap freeVariables\begin{itemize}\item{\vskip -.9ex }\end{itemize}
}
\item{
public String introducedVariable\begin{itemize}\item{\vskip -.9ex }\end{itemize}
}
\end{itemize}
}
}
\startsection{Class}{RulesOfInference}{l2}{%
{\small Contains methods for the rules of inference of first-order logic}
\vskip .1in 
\startsubsubsection{Declaration}{
\fbox{\vbox{
\hbox{\vbox{\small public final 
class 
RulesOfInference}}
\noindent\hbox{\vbox{{\bf extends} java.lang.Object}}
}}}
\startsubsubsection{Constructors}{
\vskip -2em
\begin{itemize}
\item{\vskip -1.9ex 
\membername{RulesOfInference}
{\tt public {\bf RulesOfInference}(  )
\label{l21}\label{l22}}%end signature
}%end item
\end{itemize}
}
\startsubsubsection{Methods}{
\vskip -2em
\begin{itemize}
\item{\vskip -1.9ex 
\membername{andElimination1}
{\tt public static SimpleNode {\bf andElimination1}( {\tt logic.proof.builder.parser.SimpleNode } {\bf premise} )
\label{l23}\label{l24}}%end signature
\begin{itemize}
\sld
\item{
\sld
{\bf Parameters}
\sld\isep
  \begin{itemize}
\sld\isep
   \item{
\sld
{\tt premise} - The root node of a conjunctive sentence of FOL}
  \end{itemize}
}%end item
\item{{\bf Returns} - 
leftChild The left child node of the given conjunction 
}%end item
\item{{\bf Exceptions}
  \begin{itemize}
\sld
   \item{\vskip -.6ex{\tt logic.proof.builder.exceptions.PremiseException} - If premises are not of the correct form for this rule}
  \end{itemize}
}%end item
\end{itemize}
}%end item
\divideents{andElimination1}
\item{\vskip -1.9ex 
\membername{andElimination1}
{\tt public static void {\bf andElimination1}( {\tt logic.proof.builder.parser.SimpleNode } {\bf premise},
{\tt logic.proof.builder.parser.SimpleNode } {\bf conclusion} )
\label{l25}\label{l26}}%end signature
\begin{itemize}
\sld
\item{
\sld
{\bf Parameters}
\sld\isep
  \begin{itemize}
\sld\isep
   \item{
\sld
{\tt premise} - The root node of a conjunction}
   \item{
\sld
{\tt conclusion} - The root node of the sentence being justified}
  \end{itemize}
}%end item
\item{{\bf Exceptions}
  \begin{itemize}
\sld
   \item{\vskip -.6ex{\tt logic.proof.builder.exceptions.ConclusionException} - If conclusion does not follow from using rule on given
             arguments}
   \item{\vskip -.6ex{\tt logic.proof.builder.exceptions.PremiseException} - If premises are not of the correct form for this rule}
  \end{itemize}
}%end item
\end{itemize}
}%end item
\divideents{andElimination2}
\item{\vskip -1.9ex 
\membername{andElimination2}
{\tt public static SimpleNode {\bf andElimination2}( {\tt logic.proof.builder.parser.SimpleNode } {\bf premise} )
\label{l27}\label{l28}}%end signature
\begin{itemize}
\sld
\item{
\sld
{\bf Parameters}
\sld\isep
  \begin{itemize}
\sld\isep
   \item{
\sld
{\tt premise} - The root node of a conjunction}
  \end{itemize}
}%end item
\item{{\bf Returns} - 
rightChild The right child node of the given conjunction 
}%end item
\item{{\bf Exceptions}
  \begin{itemize}
\sld
   \item{\vskip -.6ex{\tt logic.proof.builder.exceptions.PremiseException} - If conclusion does not follow from using rule on given
             arguments}
  \end{itemize}
}%end item
\end{itemize}
}%end item
\divideents{andElimination2}
\item{\vskip -1.9ex 
\membername{andElimination2}
{\tt public static void {\bf andElimination2}( {\tt logic.proof.builder.parser.SimpleNode } {\bf premise},
{\tt logic.proof.builder.parser.SimpleNode } {\bf conclusion} )
\label{l29}\label{l30}}%end signature
\begin{itemize}
\sld
\item{
\sld
{\bf Parameters}
\sld\isep
  \begin{itemize}
\sld\isep
   \item{
\sld
{\tt premise} - The root node of a conjunction}
   \item{
\sld
{\tt conclusion} - The root node of the sentence being justified}
  \end{itemize}
}%end item
\item{{\bf Exceptions}
  \begin{itemize}
\sld
   \item{\vskip -.6ex{\tt logic.proof.builder.exceptions.ConclusionException} - If conclusion does not follow from using rule on given
             arguments}
   \item{\vskip -.6ex{\tt logic.proof.builder.exceptions.PremiseException} - If premises are not of the correct form for this rule}
  \end{itemize}
}%end item
\end{itemize}
}%end item
\divideents{andIntroduction}
\item{\vskip -1.9ex 
\membername{andIntroduction}
{\tt public static SimpleNode {\bf andIntroduction}( {\tt logic.proof.builder.parser.SimpleNode } {\bf p},
{\tt logic.proof.builder.parser.SimpleNode } {\bf q} )
\label{l31}\label{l32}}%end signature
\begin{itemize}
\sld
\item{
\sld
{\bf Parameters}
\sld\isep
  \begin{itemize}
\sld\isep
   \item{
\sld
{\tt p} - The root node of a sentence of FOL}
   \item{
\sld
{\tt q} - The root node of a sentence of FOL}
  \end{itemize}
}%end item
\item{{\bf Returns} - 
conjunction The root node of a conjunction of the given
         parameters 
}%end item
\end{itemize}
}%end item
\divideents{andIntroduction}
\item{\vskip -1.9ex 
\membername{andIntroduction}
{\tt public static void {\bf andIntroduction}( {\tt logic.proof.builder.parser.SimpleNode } {\bf p},
{\tt logic.proof.builder.parser.SimpleNode } {\bf q},
{\tt logic.proof.builder.parser.SimpleNode } {\bf conclusion} )
\label{l33}\label{l34}}%end signature
\begin{itemize}
\sld
\item{
\sld
{\bf Parameters}
\sld\isep
  \begin{itemize}
\sld\isep
   \item{
\sld
{\tt p} - The root node of a sentence of FOL}
   \item{
\sld
{\tt q} - The root node of a sentence of FOL}
   \item{
\sld
{\tt conclusion} - The root node of the sentence being justified}
  \end{itemize}
}%end item
\item{{\bf Exceptions}
  \begin{itemize}
\sld
   \item{\vskip -.6ex{\tt logic.proof.builder.exceptions.ConclusionException} - If conclusion does not follow from using rule on given
             arguments}
  \end{itemize}
}%end item
\end{itemize}
}%end item
\divideents{bottomElimination}
\item{\vskip -1.9ex 
\membername{bottomElimination}
{\tt public static void {\bf bottomElimination}( {\tt logic.proof.builder.parser.SimpleNode } {\bf p},
{\tt logic.proof.builder.parser.SimpleNode } {\bf conclusion} )
\label{l35}\label{l36}}%end signature
\begin{itemize}
\sld
\item{
\sld
{\bf Parameters}
\sld\isep
  \begin{itemize}
\sld\isep
   \item{
\sld
{\tt premise} - Bottom only}
   \item{
\sld
{\tt conclusion} - The root node of the sentence being justified}
  \end{itemize}
}%end item
\item{{\bf Exceptions}
  \begin{itemize}
\sld
   \item{\vskip -.6ex{\tt logic.proof.builder.exceptions.PremiseException} - If premises are not of the correct form for this rule}
  \end{itemize}
}%end item
\end{itemize}
}%end item
\divideents{compareEqualsElim}
\item{\vskip -1.9ex 
\membername{compareEqualsElim}
{\tt public static boolean {\bf compareEqualsElim}( {\tt logic.proof.builder.parser.SimpleNode } {\bf a},
{\tt logic.proof.builder.parser.SimpleNode } {\bf b},
{\tt logic.proof.builder.parser.Variable } {\bf subVariable},
{\tt java.lang.String } {\bf newName} )
\label{l37}\label{l38}}%end signature
}%end item
\divideents{copy}
\item{\vskip -1.9ex 
\membername{copy}
{\tt public static boolean {\bf copy}( {\tt logic.proof.builder.parser.SimpleNode } {\bf p},
{\tt logic.proof.builder.parser.SimpleNode } {\bf conclusion} )
\label{l39}\label{l40}}%end signature
\begin{itemize}
\sld
\item{
\sld
{\bf Parameters}
\sld\isep
  \begin{itemize}
\sld\isep
   \item{
\sld
{\tt premise} - The root node of a sentence of FOL}
   \item{
\sld
{\tt conclusion} - The root node of the sentence being justified}
  \end{itemize}
}%end item
\item{{\bf Exceptions}
  \begin{itemize}
\sld
   \item{\vskip -.6ex{\tt logic.proof.builder.exceptions.PremiseException} - If premises are not of the correct form for this rule}
  \end{itemize}
}%end item
\end{itemize}
}%end item
\divideents{doubleNegationElimination}
\item{\vskip -1.9ex 
\membername{doubleNegationElimination}
{\tt public static SimpleNode {\bf doubleNegationElimination}( {\tt logic.proof.builder.parser.SimpleNode } {\bf p} )
\label{l41}\label{l42}}%end signature
\begin{itemize}
\sld
\item{
\sld
{\bf Parameters}
\sld\isep
  \begin{itemize}
\sld\isep
   \item{
\sld
{\tt p} - The root node of a sentence of FOL starting with two negations}
  \end{itemize}
}%end item
\item{{\bf Returns} - 
The premise without the first two negations 
}%end item
\item{{\bf Exceptions}
  \begin{itemize}
\sld
   \item{\vskip -.6ex{\tt logic.proof.builder.exceptions.PremiseException} - If premises are not of the correct form for this rule}
  \end{itemize}
}%end item
\end{itemize}
}%end item
\divideents{doubleNegationElimination}
\item{\vskip -1.9ex 
\membername{doubleNegationElimination}
{\tt public static void {\bf doubleNegationElimination}( {\tt logic.proof.builder.parser.SimpleNode } {\bf p},
{\tt logic.proof.builder.parser.SimpleNode } {\bf conclusion} )
\label{l43}\label{l44}}%end signature
\begin{itemize}
\sld
\item{
\sld
{\bf Parameters}
\sld\isep
  \begin{itemize}
\sld\isep
   \item{
\sld
{\tt premise} - The root node of a sentence of FOL starting with two negations}
   \item{
\sld
{\tt conclusion} - The root node of the sentence being justified}
  \end{itemize}
}%end item
\item{{\bf Exceptions}
  \begin{itemize}
\sld
   \item{\vskip -.6ex{\tt logic.proof.builder.exceptions.ConclusionException} - If conclusion does not follow from using rule on given
             arguments}
   \item{\vskip -.6ex{\tt logic.proof.builder.exceptions.PremiseException} - If premises are not of the correct form for this rule}
  \end{itemize}
}%end item
\end{itemize}
}%end item
\divideents{doubleNegationIntroduction}
\item{\vskip -1.9ex 
\membername{doubleNegationIntroduction}
{\tt public static SimpleNode {\bf doubleNegationIntroduction}( {\tt logic.proof.builder.parser.SimpleNode } {\bf p} )
\label{l45}\label{l46}}%end signature
\begin{itemize}
\sld
\item{
\sld
{\bf Parameters}
\sld\isep
  \begin{itemize}
\sld\isep
   \item{
\sld
{\tt p} - The root node of a sentence of FOL}
  \end{itemize}
}%end item
\item{{\bf Returns} - 
the premise with two negations appended to the start 
}%end item
\end{itemize}
}%end item
\divideents{doubleNegationIntroduction}
\item{\vskip -1.9ex 
\membername{doubleNegationIntroduction}
{\tt public static void {\bf doubleNegationIntroduction}( {\tt logic.proof.builder.parser.SimpleNode } {\bf p},
{\tt logic.proof.builder.parser.SimpleNode } {\bf conclusion} )
\label{l47}\label{l48}}%end signature
\begin{itemize}
\sld
\item{
\sld
{\bf Parameters}
\sld\isep
  \begin{itemize}
\sld\isep
   \item{
\sld
{\tt p} - The root node of a sentence of FOL}
   \item{
\sld
{\tt conclusion} - The root node of the sentence being justified, should start
            with double negation}
  \end{itemize}
}%end item
\item{{\bf Exceptions}
  \begin{itemize}
\sld
   \item{\vskip -.6ex{\tt logic.proof.builder.exceptions.ConclusionException} - If conclusion does not follow from using rule on given
             arguments}
  \end{itemize}
}%end item
\end{itemize}
}%end item
\divideents{equalsElimination}
\item{\vskip -1.9ex 
\membername{equalsElimination}
{\tt public static void {\bf equalsElimination}( {\tt logic.proof.builder.parser.SimpleNode } {\bf equals},
{\tt logic.proof.builder.parser.SimpleNode } {\bf statement},
{\tt logic.proof.builder.parser.Variable } {\bf variable},
{\tt logic.proof.builder.parser.SimpleNode } {\bf conclusion} )
\label{l49}\label{l50}}%end signature
\begin{itemize}
\sld
\item{
\sld
{\bf Parameters}
\sld\isep
  \begin{itemize}
\sld\isep
   \item{
\sld
{\tt equals} - A sentence of the form t1 = t2}
   \item{
\sld
{\tt statement} - A sentence of FOL, should contain t1}
   \item{
\sld
{\tt variable} - The free variable t1}
   \item{
\sld
{\tt conclusion} - The root node of the sentence being justified}
  \end{itemize}
}%end item
\item{{\bf Exceptions}
  \begin{itemize}
\sld
   \item{\vskip -.6ex{\tt logic.proof.builder.exceptions.ConclusionException} - If conclusion does not follow from using rule on given
             arguments}
   \item{\vskip -.6ex{\tt logic.proof.builder.exceptions.PremiseException} - If premises are not of the correct form for this rule}
  \end{itemize}
}%end item
\end{itemize}
}%end item
\divideents{equalsIntroduction}
\item{\vskip -1.9ex 
\membername{equalsIntroduction}
{\tt public static void {\bf equalsIntroduction}( {\tt logic.proof.builder.parser.SimpleNode } {\bf conclusion} )
\label{l51}\label{l52}}%end signature
\begin{itemize}
\sld
\item{
\sld
{\bf Parameters}
\sld\isep
  \begin{itemize}
\sld\isep
   \item{
\sld
{\tt conclusion} - The root node of the sentence being justified, must have the
            form t = t}
  \end{itemize}
}%end item
\item{{\bf Exceptions}
  \begin{itemize}
\sld
   \item{\vskip -.6ex{\tt logic.proof.builder.exceptions.ConclusionException} - If conclusion does not follow from using rule on given
             arguments}
  \end{itemize}
}%end item
\end{itemize}
}%end item
\divideents{existsElimination}
\item{\vskip -1.9ex 
\membername{existsElimination}
{\tt public static void {\bf existsElimination}( {\tt logic.proof.builder.parser.SimpleNode } {\bf p},
{\tt java.util.ArrayList } {\bf subproof},
{\tt java.lang.String } {\bf variableName},
{\tt logic.proof.builder.parser.SimpleNode } {\bf conclusion} )
\label{l53}\label{l54}}%end signature
\begin{itemize}
\sld
\item{
\sld
{\bf Parameters}
\sld\isep
  \begin{itemize}
\sld\isep
   \item{
\sld
{\tt p} - The root node of a sentence of FOL, should be existentially
            quantified}
   \item{
\sld
{\tt subproof} - a Subproof starting with a sentence in which the existentially quantified variable of the premise is named and ends in any sentence of FOL}
   \item{
\sld
{\tt variableName} - The name of the variable that is introduced}
   \item{
\sld
{\tt conclusion} - The root node of the sentence being justified}
  \end{itemize}
}%end item
\item{{\bf Exceptions}
  \begin{itemize}
\sld
   \item{\vskip -.6ex{\tt logic.proof.builder.exceptions.PremiseException} - If premises are not of the correct form for this rule}
   \item{\vskip -.6ex{\tt logic.proof.builder.exceptions.ConclusionException} - If conclusion does not follow from using rule on given
             arguments}
  \end{itemize}
}%end item
\end{itemize}
}%end item
\divideents{existsIntroduction}
\item{\vskip -1.9ex 
\membername{existsIntroduction}
{\tt public static void {\bf existsIntroduction}( {\tt logic.proof.builder.parser.SimpleNode } {\bf p},
{\tt logic.proof.builder.parser.SimpleNode } {\bf conclusion} )
\label{l55}\label{l56}}%end signature
\begin{itemize}
\sld
\item{
\sld
{\bf Parameters}
\sld\isep
  \begin{itemize}
\sld\isep
   \item{
\sld
{\tt premise} - The root node of a sentence of FOL}
   \item{
\sld
{\tt conclusion} - The root node of the sentence being justified, should begin
            with an existential quantifer}
  \end{itemize}
}%end item
\item{{\bf Exceptions}
  \begin{itemize}
\sld
   \item{\vskip -.6ex{\tt logic.proof.builder.exceptions.ConclusionException} - If conclusion does not follow from using rule on given
             arguments}
  \end{itemize}
}%end item
\end{itemize}
}%end item
\divideents{forAllElimination}
\item{\vskip -1.9ex 
\membername{forAllElimination}
{\tt public static void {\bf forAllElimination}( {\tt logic.proof.builder.parser.SimpleNode } {\bf forAll},
{\tt logic.proof.builder.parser.SimpleNode } {\bf conclusion} )
\label{l57}\label{l58}}%end signature
\begin{itemize}
\sld
\item{
\sld
{\bf Parameters}
\sld\isep
  \begin{itemize}
\sld\isep
   \item{
\sld
{\tt forAll} - The root node of universally quantified sentence of FOL}
   \item{
\sld
{\tt conclusion} - The root node of the sentence being justified, should be the
            unquantified version of the premise}
  \end{itemize}
}%end item
\item{{\bf Exceptions}
  \begin{itemize}
\sld
   \item{\vskip -.6ex{\tt logic.proof.builder.exceptions.PremiseException} - If premises are not of the correct form for this rule}
   \item{\vskip -.6ex{\tt logic.proof.builder.exceptions.ConclusionException} - If conclusion does not follow from using rule on given
             arguments}
  \end{itemize}
}%end item
\end{itemize}
}%end item
\divideents{forAllIntroduction}
\item{\vskip -1.9ex 
\membername{forAllIntroduction}
{\tt public static void {\bf forAllIntroduction}( {\tt logic.proof.builder.parser.SimpleNode } {\bf p},
{\tt logic.proof.builder.parser.Variable } {\bf variable},
{\tt logic.proof.builder.parser.SimpleNode } {\bf conclusion} )
\label{l59}\label{l60}}%end signature
\begin{itemize}
\sld
\item{
\sld
{\bf Parameters}
\sld\isep
  \begin{itemize}
\sld\isep
   \item{
\sld
{\tt p} - The final line of the subproof, should contain the introduced
            variable}
   \item{
\sld
{\tt variable} - The introduced variable}
   \item{
\sld
{\tt conclusion} - The universally quantified version of the premise}
  \end{itemize}
}%end item
\item{{\bf Exceptions}
  \begin{itemize}
\sld
   \item{\vskip -.6ex{\tt logic.proof.builder.exceptions.ConclusionException} - If conclusion does not follow from using rule on given
             arguments}
  \end{itemize}
}%end item
\end{itemize}
}%end item
\divideents{impliesIntroduction}
\item{\vskip -1.9ex 
\membername{impliesIntroduction}
{\tt public static SimpleNode {\bf impliesIntroduction}( {\tt java.util.List } {\bf subproof} )
\label{l61}\label{l62}}%end signature
\begin{itemize}
\sld
\item{
\sld
{\bf Parameters}
\sld\isep
  \begin{itemize}
\sld\isep
   \item{
\sld
{\tt subproof} - Any subproof}
  \end{itemize}
}%end item
\item{{\bf Returns} - 
an implication where the LHS is the first line of the given subproof and the RHS is the last line 
}%end item
\end{itemize}
}%end item
\divideents{impliesIntroduction}
\item{\vskip -1.9ex 
\membername{impliesIntroduction}
{\tt public static void {\bf impliesIntroduction}( {\tt java.util.List } {\bf subproof},
{\tt logic.proof.builder.parser.SimpleNode } {\bf conclusion} )
\label{l63}\label{l64}}%end signature
\begin{itemize}
\sld
\item{
\sld
{\bf Parameters}
\sld\isep
  \begin{itemize}
\sld\isep
   \item{
\sld
{\tt premise} - The root node of a sentence of FOL}
   \item{
\sld
{\tt conclusion} - The root node of the sentence being justified}
  \end{itemize}
}%end item
\item{{\bf Exceptions}
  \begin{itemize}
\sld
   \item{\vskip -.6ex{\tt logic.proof.builder.exceptions.ConclusionException} - If conclusion does not follow from using rule on given
             arguments}
  \end{itemize}
}%end item
\end{itemize}
}%end item
\divideents{modusPonens}
\item{\vskip -1.9ex 
\membername{modusPonens}
{\tt public static SimpleNode {\bf modusPonens}( {\tt logic.proof.builder.parser.SimpleNode } {\bf p},
{\tt logic.proof.builder.parser.SimpleNode } {\bf implication} )
\label{l65}\label{l66}}%end signature
\begin{itemize}
\sld
\item{
\sld
{\bf Parameters}
\sld\isep
  \begin{itemize}
\sld\isep
   \item{
\sld
{\tt p} - The root node of a sentence of FOL}
   \item{
\sld
{\tt implication} - The root node of a material implication sentence. The LHS
            should be the previous argument.}
  \end{itemize}
}%end item
\item{{\bf Returns} - 
the RHS of the implication 
}%end item
\item{{\bf Exceptions}
  \begin{itemize}
\sld
   \item{\vskip -.6ex{\tt logic.proof.builder.exceptions.PremiseException} - If premises are not of the correct form for this rule}
  \end{itemize}
}%end item
\end{itemize}
}%end item
\divideents{modusPonens}
\item{\vskip -1.9ex 
\membername{modusPonens}
{\tt public static void {\bf modusPonens}( {\tt logic.proof.builder.parser.SimpleNode } {\bf p},
{\tt logic.proof.builder.parser.SimpleNode } {\bf implication},
{\tt logic.proof.builder.parser.SimpleNode } {\bf conclusion} )
\label{l67}\label{l68}}%end signature
\begin{itemize}
\sld
\item{
\sld
{\bf Parameters}
\sld\isep
  \begin{itemize}
\sld\isep
   \item{
\sld
{\tt p} - The root node of a sentence of FOL}
   \item{
\sld
{\tt implication} - The root node of a material implication sentence. The LHS
            should be the previous argument.}
   \item{
\sld
{\tt conclusion} - The root node of the sentence being justified}
  \end{itemize}
}%end item
\item{{\bf Exceptions}
  \begin{itemize}
\sld
   \item{\vskip -.6ex{\tt logic.proof.builder.exceptions.ConclusionException} - If conclusion does not follow from using rule on given
             arguments}
   \item{\vskip -.6ex{\tt logic.proof.builder.exceptions.PremiseException} - If premises are not of the correct form for this rule}
  \end{itemize}
}%end item
\end{itemize}
}%end item
\divideents{negationElimination}
\item{\vskip -1.9ex 
\membername{negationElimination}
{\tt public static SimpleNode {\bf negationElimination}( {\tt logic.proof.builder.parser.SimpleNode } {\bf p},
{\tt logic.proof.builder.parser.SimpleNode } {\bf notP} )
\label{l69}\label{l70}}%end signature
\begin{itemize}
\sld
\item{
\sld
{\bf Parameters}
\sld\isep
  \begin{itemize}
\sld\isep
   \item{
\sld
{\tt p} - The root node of a sentence of FOL}
   \item{
\sld
{\tt notP} - The negation of the previous argument}
  \end{itemize}
}%end item
\item{{\bf Returns} - 
bottom 
}%end item
\item{{\bf Exceptions}
  \begin{itemize}
\sld
   \item{\vskip -.6ex{\tt logic.proof.builder.exceptions.PremiseException} - If premises are not of the correct form for this rule}
  \end{itemize}
}%end item
\end{itemize}
}%end item
\divideents{negationElimination}
\item{\vskip -1.9ex 
\membername{negationElimination}
{\tt public static void {\bf negationElimination}( {\tt logic.proof.builder.parser.SimpleNode } {\bf p},
{\tt logic.proof.builder.parser.SimpleNode } {\bf notP},
{\tt logic.proof.builder.parser.SimpleNode } {\bf conclusion} )
\label{l71}\label{l72}}%end signature
\begin{itemize}
\sld
\item{
\sld
{\bf Parameters}
\sld\isep
  \begin{itemize}
\sld\isep
   \item{
\sld
{\tt p} - The root node of a sentence of FOL}
   \item{
\sld
{\tt notP} - The negation of the previous argument}
   \item{
\sld
{\tt conclusion} - The root node of the sentence being justified, should only be
            bottom}
  \end{itemize}
}%end item
\item{{\bf Exceptions}
  \begin{itemize}
\sld
   \item{\vskip -.6ex{\tt logic.proof.builder.exceptions.PremiseException} - If premises are not of the correct form for this rule}
   \item{\vskip -.6ex{\tt logic.proof.builder.exceptions.ConclusionException} - If conclusion does not follow from using rule on given
             arguments}
  \end{itemize}
}%end item
\end{itemize}
}%end item
\divideents{negationIntroduction}
\item{\vskip -1.9ex 
\membername{negationIntroduction}
{\tt public static SimpleNode {\bf negationIntroduction}( {\tt java.util.List } {\bf subproof} )
\label{l73}\label{l74}}%end signature
\begin{itemize}
\sld
\item{
\sld
{\bf Parameters}
\sld\isep
  \begin{itemize}
\sld\isep
   \item{
\sld
{\tt subproof} - A list of proofsteps ending with bottom}
  \end{itemize}
}%end item
\item{{\bf Returns} - 
the negation of the first line of the subproof 
}%end item
\item{{\bf Exceptions}
  \begin{itemize}
\sld
   \item{\vskip -.6ex{\tt logic.proof.builder.exceptions.PremiseException} - If premises are not of the correct form for this rule}
  \end{itemize}
}%end item
\end{itemize}
}%end item
\divideents{negationIntroduction}
\item{\vskip -1.9ex 
\membername{negationIntroduction}
{\tt public static void {\bf negationIntroduction}( {\tt java.util.List } {\bf subproof},
{\tt logic.proof.builder.parser.SimpleNode } {\bf conclusion} )
\label{l75}\label{l76}}%end signature
\begin{itemize}
\sld
\item{
\sld
{\bf Parameters}
\sld\isep
  \begin{itemize}
\sld\isep
   \item{
\sld
{\tt premise} - The root node of a sentence of FOL}
   \item{
\sld
{\tt conclusion} - The root node of the sentence being justified, should start
            with a negation}
  \end{itemize}
}%end item
\item{{\bf Exceptions}
  \begin{itemize}
\sld
   \item{\vskip -.6ex{\tt logic.proof.builder.exceptions.PremiseException} - If premises are not of the correct form for this rule}
   \item{\vskip -.6ex{\tt logic.proof.builder.exceptions.ConclusionException} - If conclusion does not follow from using rule on given
             arguments}
  \end{itemize}
}%end item
\end{itemize}
}%end item
\divideents{orElimination}
\item{\vskip -1.9ex 
\membername{orElimination}
{\tt public static SimpleNode {\bf orElimination}( {\tt java.util.List } {\bf subproof1},
{\tt java.util.List } {\bf subproof2},
{\tt logic.proof.builder.parser.SimpleNode } {\bf disjunction} )
\label{l77}\label{l78}}%end signature
\begin{itemize}
\sld
\item{
\sld
{\bf Parameters}
\sld\isep
  \begin{itemize}
\sld\isep
   \item{
\sld
{\tt subproof1} - A subproof starting with the LHS of the disjunction}
   \item{
\sld
{\tt subproof2} - A subproof starting with the RHS of the disjunction and ending with the same sentence as the previous subproof}
   \item{
\sld
{\tt disjunction} - The root node of a disjunction of sentences}
  \end{itemize}
}%end item
\item{{\bf Returns} - 
chi The root node of the sentence that both subprrofs end with 
}%end item
\item{{\bf Exceptions}
  \begin{itemize}
\sld
   \item{\vskip -.6ex{\tt logic.proof.builder.exceptions.PremiseException} - If premises are not of the correct form for this rule}
  \end{itemize}
}%end item
\end{itemize}
}%end item
\divideents{orElimination}
\item{\vskip -1.9ex 
\membername{orElimination}
{\tt public static void {\bf orElimination}( {\tt java.util.List } {\bf subproof1},
{\tt java.util.List } {\bf subproof2},
{\tt logic.proof.builder.parser.SimpleNode } {\bf disjunction},
{\tt logic.proof.builder.parser.SimpleNode } {\bf conclusion} )
\label{l79}\label{l80}}%end signature
\begin{itemize}
\sld
\item{
\sld
{\bf Parameters}
\sld\isep
  \begin{itemize}
\sld\isep
   \item{
\sld
{\tt subproof1} - A list of}
   \item{
\sld
{\tt subproof2} - The root node of the sentence being justified}
   \item{
\sld
{\tt disjunction} - The root node of a disjunction of sentences}
   \item{
\sld
{\tt conclusion} - The root node of the sentence being justified}
  \end{itemize}
}%end item
\item{{\bf Exceptions}
  \begin{itemize}
\sld
   \item{\vskip -.6ex{\tt logic.proof.builder.exceptions.ConclusionException} - If conclusion does not follow from using rule on given
             arguments}
   \item{\vskip -.6ex{\tt logic.proof.builder.exceptions.PremiseException} - If premises are not of the correct form for this rule}
  \end{itemize}
}%end item
\end{itemize}
}%end item
\divideents{orIntroduction1}
\item{\vskip -1.9ex 
\membername{orIntroduction1}
{\tt public static void {\bf orIntroduction1}( {\tt logic.proof.builder.parser.SimpleNode } {\bf premise},
{\tt logic.proof.builder.parser.SimpleNode } {\bf conclusion} )
\label{l81}\label{l82}}%end signature
\begin{itemize}
\sld
\item{
\sld
{\bf Parameters}
\sld\isep
  \begin{itemize}
\sld\isep
   \item{
\sld
{\tt premise} - The root node of a sentence of FOL}
   \item{
\sld
{\tt conclusion} - The root node of the sentence being justified, must be a
            disjunctive sentence}
  \end{itemize}
}%end item
\item{{\bf Exceptions}
  \begin{itemize}
\sld
   \item{\vskip -.6ex{\tt logic.proof.builder.exceptions.ConclusionException} - If conclusion does not follow from using rule on given
             arguments}
   \item{\vskip -.6ex{\tt logic.proof.builder.exceptions.PremiseException} - If premises are not of the correct form for this rule}
  \end{itemize}
}%end item
\end{itemize}
}%end item
\divideents{orIntroduction2}
\item{\vskip -1.9ex 
\membername{orIntroduction2}
{\tt public static void {\bf orIntroduction2}( {\tt logic.proof.builder.parser.SimpleNode } {\bf premise},
{\tt logic.proof.builder.parser.SimpleNode } {\bf conclusion} )
\label{l83}\label{l84}}%end signature
\begin{itemize}
\sld
\item{
\sld
{\bf Parameters}
\sld\isep
  \begin{itemize}
\sld\isep
   \item{
\sld
{\tt premise} - The root node of a sentence of FOL}
   \item{
\sld
{\tt conclusion} - The root node of the sentence being justified, must be a
            disjunctive sentence}
  \end{itemize}
}%end item
\item{{\bf Exceptions}
  \begin{itemize}
\sld
   \item{\vskip -.6ex{\tt logic.proof.builder.exceptions.ConclusionException} - If conclusion does not follow from using rule on given
             arguments}
   \item{\vskip -.6ex{\tt logic.proof.builder.exceptions.PremiseException} - If premises are not of the correct form for this rule}
  \end{itemize}
}%end item
\end{itemize}
}%end item
\end{itemize}
}
}
}
}
\end{document}
