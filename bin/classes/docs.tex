\documentclass[11pt]{report}
\def\bl{\mbox{}\newline\mbox{}\newline{}}
\usepackage{ifthen}
\newcommand{\hide}[2]{
\ifthenelse{\equal{#1}{inherited}}%
{}%
{}%
}
\newcommand{\entityintro}[3]{%
  \hbox to \hsize{%
    \vbox{%
      \hbox to .2in{}%
    }%
    {\bf #1}%
    \dotfill\pageref{#2}%
  }
  \makebox[\hsize]{%
    \parbox{.4in}{}%
    \parbox[l]{5in}{%
      \vspace{1mm}\it%
      #3%
      \vspace{1mm}%
    }%
  }%
}
\newcommand{\isep}[0]{%
\setlength{\itemsep}{-.4ex}
}
\newcommand{\sld}[0]{%
\setlength{\topsep}{0em}
\setlength{\partopsep}{0em}
\setlength{\parskip}{0em}
\setlength{\parsep}{-1em}
}
\newcommand{\headref}[3]{%
\ifthenelse{#1 = 1}{%
\addcontentsline{toc}{section}{\hspace{\qquad}\protect\numberline{}{#3}}%
}{}%
\ifthenelse{#1 = 2}{%
\addcontentsline{toc}{subsection}{\hspace{\qquad}\protect\numerline{}{#3}}%
}{}%
\ifthenelse{#1 = 3}{%
\addcontentsline{toc}{subsubsection}{\hspace{\qquad}\protect\numerline{}{#3}}%
}{}%
\label{#3}%
\makebox[\textwidth][l]{#2 #3}%
}%
\newcommand{\membername}[1]{{\it #1}\linebreak}
\newcommand{\divideents}[1]{\vskip -1em\indent\rule{2in}{.5mm}}
\newcommand{\refdefined}[1]{
\expandafter\ifx\csname r@#1\endcsname\relax
\relax\else
{$($ in \ref{#1}, page \pageref{#1}$)$}
\fi}
\newcommand{\startsection}[4]{
\gdef\classname{#2}
\subsection{\label{#3}{\bf {\sc #1} #2}}{
\rule[1em]{\hsize}{4pt}\vskip -1em
\vskip .1in 
#4
}%
}
\newcommand{\startsubsubsection}[2]{
\subsubsection{\sc #1}{%
\rule[1em]{\hsize}{2pt}%
#2}
}
\usepackage{color}
\date{\today}
\pagestyle{myheadings}
\addtocontents{toc}{\protect\def\protect\packagename{}}
\addtocontents{toc}{\protect\def\protect\classname{}}
\markboth{\protect\packagename -- \protect\classname}{\protect\packagename -- \protect\classname}
\oddsidemargin 0in
\evensidemargin 0in
% \topmargin -.8in
\chardef\bslash=`\\
\textheight 9.4in
\textwidth 6.5in
\begin{document}
\sloppy
\raggedright
\tableofcontents
\gdef\packagename{}
\gdef\classname{}
\newpage
\def\packagename{logic.proof.builder.proof}
\chapter{\bf Package logic.proof.builder.proof}{
\vskip -.25in
\hbox to \hsize{\it Package Contents\hfil Page}
\rule{\hsize}{.7mm}
\vskip .13in
\hbox{\bf Classes}
\entityintro{Proof}{l0}{Stores all data necessary to construct a proof.}
\entityintro{ProofStep}{l1}{...no description...}
\entityintro{RulesOfInference}{l2}{Contains methods for the rules of inference of first-order logic}
\vskip .1in
\rule{\hsize}{.7mm}
\vskip .1in
\newpage
\section{Classes}{
\startsection{Class}{Proof}{l0}{%
{\small Stores all data necessary to construct a proof. Simple Methods are provided to manipulate
 the proof such as adding or deleting lines.}
\vskip .1in 
\startsubsubsection{Declaration}{
\fbox{\vbox{
\hbox{\vbox{\small public 
class 
Proof}}
\noindent\hbox{\vbox{{\bf extends} java.lang.Object}}
}}}
\startsubsubsection{Fields}{
\begin{itemize}
\item{
public List predicates\begin{itemize}\item{\vskip -.9ex A list of all the named predicates in the proof. Used to populate the
 predicate list.}\end{itemize}
}
\end{itemize}
}
\startsubsubsection{Constructors}{
\vskip -2em
\begin{itemize}
\item{\vskip -1.9ex 
\membername{Proof}
{\tt public {\bf Proof}(  )
\label{l3}\label{l4}}%end signature
\begin{itemize}
\sld
\item{
\sld
{\bf Usage}
  \begin{itemize}\isep
   \item{
Default constructor. Constructs an empty proof
}%end item
  \end{itemize}
}
\end{itemize}
}%end item
\end{itemize}
}
\startsubsubsection{Methods}{
\vskip -2em
\begin{itemize}
\item{\vskip -1.9ex 
\membername{addStepAsEndOfSubproof}
{\tt public ProofStep {\bf addStepAsEndOfSubproof}( {\tt logic.proof.builder.parser.SimpleNode } {\bf node},
{\tt java.lang.String } {\bf formula} )
\label{l5}\label{l6}}%end signature
\begin{itemize}
\sld
\item{
\sld
{\bf Usage}
  \begin{itemize}\isep
   \item{
Adds a new proofstep that is the last line of a subproof
}%end item
  \end{itemize}
}
\item{
\sld
{\bf Parameters}
\sld\isep
  \begin{itemize}
\sld\isep
   \item{
\sld
{\tt node} - Root node of the sentence of the proofstep}
   \item{
\sld
{\tt formula} - String representation of the sentence}
  \end{itemize}
}%end item
\item{{\bf Returns} - 
Returns the proofstep that has been added 
}%end item
\end{itemize}
}%end item
\divideents{addStepAsNewLine}
\item{\vskip -1.9ex 
\membername{addStepAsNewLine}
{\tt public ProofStep {\bf addStepAsNewLine}( {\tt logic.proof.builder.parser.SimpleNode } {\bf node},
{\tt java.lang.String } {\bf formula} )
\label{l7}\label{l8}}%end signature
\begin{itemize}
\sld
\item{
\sld
{\bf Usage}
  \begin{itemize}\isep
   \item{
The default method to add a new proofstep to the proof
}%end item
  \end{itemize}
}
\item{
\sld
{\bf Parameters}
\sld\isep
  \begin{itemize}
\sld\isep
   \item{
\sld
{\tt node} - Root node of the sentence of the proofstep}
   \item{
\sld
{\tt formula} - String representation of the sentence}
  \end{itemize}
}%end item
\item{{\bf Returns} - 
Returns the proofstep that has been added 
}%end item
\end{itemize}
}%end item
\divideents{addStepAsStartOfSubproof}
\item{\vskip -1.9ex 
\membername{addStepAsStartOfSubproof}
{\tt public ProofStep {\bf addStepAsStartOfSubproof}( {\tt logic.proof.builder.parser.SimpleNode } {\bf node},
{\tt java.lang.String } {\bf formula} )
\label{l9}\label{l10}}%end signature
\begin{itemize}
\sld
\item{
\sld
{\bf Usage}
  \begin{itemize}\isep
   \item{
Adds a new proofstep that is the start of a subproof
}%end item
  \end{itemize}
}
\item{
\sld
{\bf Parameters}
\sld\isep
  \begin{itemize}
\sld\isep
   \item{
\sld
{\tt node} - Root node of the sentence of the proofstep}
   \item{
\sld
{\tt formula} - String representation of the sentence}
  \end{itemize}
}%end item
\item{{\bf Returns} - 
Returns the proofstep that has been added 
}%end item
\end{itemize}
}%end item
\divideents{addVar}
\item{\vskip -1.9ex 
\membername{addVar}
{\tt public ProofStep {\bf addVar}( {\tt java.lang.String } {\bf var} )
\label{l11}\label{l12}}%end signature
\begin{itemize}
\sld
\item{
\sld
{\bf Usage}
  \begin{itemize}\isep
   \item{
Add a new proofstep which introduces a boxed variable
}%end item
  \end{itemize}
}
\item{
\sld
{\bf Parameters}
\sld\isep
  \begin{itemize}
\sld\isep
   \item{
\sld
{\tt var} - The name of the variable being introduced}
  \end{itemize}
}%end item
\item{{\bf Returns} - 
Returns the proofstep that has been added 
}%end item
\end{itemize}
}%end item
\divideents{addVar}
\item{\vskip -1.9ex 
\membername{addVar}
{\tt public ProofStep {\bf addVar}( {\tt java.lang.String } {\bf introducedVariable},
{\tt logic.proof.builder.parser.SimpleNode } {\bf rootNode},
{\tt java.lang.String } {\bf formula} )
\label{l13}\label{l14}}%end signature
\begin{itemize}
\sld
\item{
\sld
{\bf Usage}
  \begin{itemize}\isep
   \item{
Add a new proofstep which introduces a boxed variable alongside an
 assumption
}%end item
  \end{itemize}
}
\item{
\sld
{\bf Parameters}
\sld\isep
  \begin{itemize}
\sld\isep
   \item{
\sld
{\tt introducedVariable} - The name of the variable being introduced}
   \item{
\sld
{\tt node} - Root node of the sentence}
   \item{
\sld
{\tt formula} - String representation of the sentence}
  \end{itemize}
}%end item
\item{{\bf Returns} - 
Returns the proofstep that has been added 
}%end item
\end{itemize}
}%end item
\divideents{getCurrentLevel}
\item{\vskip -1.9ex 
\membername{getCurrentLevel}
{\tt public int {\bf getCurrentLevel}(  )
\label{l15}\label{l16}}%end signature
\begin{itemize}
\sld
\item{
\sld
{\bf Usage}
  \begin{itemize}\isep
   \item{
Returns the number of subproofs currently open
}%end item
  \end{itemize}
}
\item{{\bf Returns} - 
the number of subproofs currently open 
}%end item
\end{itemize}
}%end item
\divideents{getLines}
\item{\vskip -1.9ex 
\membername{getLines}
{\tt public ArrayList {\bf getLines}(  )
\label{l17}\label{l18}}%end signature
\begin{itemize}
\sld
\item{
\sld
{\bf Usage}
  \begin{itemize}\isep
   \item{
Returns the ordered list of proofsteps
}%end item
  \end{itemize}
}
\item{{\bf Returns} - 
the ordered list of proofsteps 
}%end item
\end{itemize}
}%end item
\divideents{removeStep}
\item{\vskip -1.9ex 
\membername{removeStep}
{\tt public void {\bf removeStep}(  )
\label{l19}\label{l20}}%end signature
\begin{itemize}
\sld
\item{
\sld
{\bf Usage}
  \begin{itemize}\isep
   \item{
Removes the most recent line from the proof
}%end item
  \end{itemize}
}
\end{itemize}
}%end item
\end{itemize}
}
}
\startsection{Class}{ProofStep}{l1}{%
\startsubsubsection{Declaration}{
\fbox{\vbox{
\hbox{\vbox{\small public 
class 
ProofStep}}
\noindent\hbox{\vbox{{\bf extends} java.lang.Object}}
}}}
\startsubsubsection{Fields}{
\begin{itemize}
\item{
public ProofStep parent\begin{itemize}\item{\vskip -.9ex }\end{itemize}
}
\item{
public List subproofs\begin{itemize}\item{\vskip -.9ex }\end{itemize}
}
\item{
public ProofStep next\begin{itemize}\item{\vskip -.9ex }\end{itemize}
}
\item{
public SimpleNode node\begin{itemize}\item{\vskip -.9ex }\end{itemize}
}
\item{
public Integer lineNumber\begin{itemize}\item{\vskip -.9ex }\end{itemize}
}
\item{
public int level\begin{itemize}\item{\vskip -.9ex }\end{itemize}
}
\item{
public String formula\begin{itemize}\item{\vskip -.9ex }\end{itemize}
}
\item{
public String justification\begin{itemize}\item{\vskip -.9ex }\end{itemize}
}
\item{
public boolean endOfSubproof\begin{itemize}\item{\vskip -.9ex }\end{itemize}
}
\item{
public HashMap freeVariables\begin{itemize}\item{\vskip -.9ex }\end{itemize}
}
\item{
public String introducedVariable\begin{itemize}\item{\vskip -.9ex }\end{itemize}
}
\end{itemize}
}
}

\vskip -2em

}

}
\end{document}
