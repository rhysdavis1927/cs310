\nocite{*}
\section{Background}

Logic can be used to refer to both the use of valid reasoning and as a distinctly different entity, the study of valid reasoning. It is the latter of these that will be the main focus of our attention. As such, the use of logic will subsequently be used to not simply mean the use of valid argument, but rather a reflection upon the principles of validity. While the use of valid reasoning has been attributed to early humans and even to non-human animals \cite{wolfgang1999mentality}, the earliest known systematic study of formal logic is attributed to Aristotle in the 4th century BC. \cite[p.~300]{salmon2007introduction}

The works of Aristotle, later grouped into in a collection known as the Organon, drew from sources such as the works of Aristotle’s teacher Plato, Zeno of Elea and geometrical proofs. \cite{KingShapiro95} 

One of Aristotle's most influential pieces was that of the syllogism which Aristotle defined as ``a discourse in which, certain things being stated, something other than what is stated follows of necessity from their being so.'' \cite[p.~4]{aristotle2004prior} Aristotle generally restricted a syllogism to three parts: the major premise, the minor premise and the conclusion. The premises were restricted to the form it where it contains a subject and a predicate and must either affirm or deny the predicate of the subject. \cite{sep-aristotle-logic}
-explain terms


It is believed that the Pythagorean schools were the first to have attempted a systematic study of the demonstrative sciences and Aristotle used some of the ideas from the deductive system used in geometrical proofs. \cite[p.~3]{kneale1962developments}

 An important concept in Aristotle's deductive system was that of Reductio ad Impossibile, which demonstrates that an assumption is false by showing that the original assumption leads to a contradictory conclusion. Aristotle attributed this concept to Zeno of Elea but compares it to the use Reductio ad Absurdum as used by Euclid in the proof that the square root of 2 is irrational.\cite[p.~8]{kneale1962developments}

Reductio ad Absurdum demonstrates that an assumption is false by showing that the original assumption leads to drawing of false conclusions. While Reductio ad Impossibile is similar to Reductio ad Absurdum, the latter is stronger since it only requires that the conclusion is clearly false, not necessarily a contradiction. \cite[p.~9]{kneale1962developments}



The term is a part of speech representing something, but which is not true or false in its own right, such as "man" or "mortal".
The proposition consists of two terms, in which one term (the "predicate") is "affirmed" or "denied" of the other (the "subject"), and which is capable of truth or falsity.

The Stoic school developed a parallel logical system to Aristotelian logic. This system was focused on prepositions rather than terms. The greatest contributor of the Stoics were made by Chrysippus. Chrysippus seperated prepositions into two classes, these classes are the equivalent of the modern day atomic and compostie propositions. The Stoics became the first of the logicians to study conditional statements. These conditional statements included strict implication, relevant implication and the now well know material implication i.e. statements of the from p $\Rightarrow$ q or equivalently if p then q, where p and q are propositions. They discussed the truth values of these conditionals and other logical connectives. From these discussions it is possible to construct the modern day truth table representation of these connectives. 

Despite the belief, at the time of creation, that these two rivals were incompatible; the two systems were actually complimentary and eventually merged into one logical system.

After this period few major developments were made in western logic for quite some time. Most of the work involved simplification and combination of previous resulting in greater clarity. An effect of this greater clarity was that certain some concepts such as relevance implication were rejected. \cite{KingShapiro95} 





In the 18th century Leibniz, inspired by mathematics, envisaged a symbolic logic which would enable a form of logical calculus. Such a system would eventually come to fruition, however, this system was developed independently of Leibniz's ideas despite his early predictions. \cite{sep-leibniz-logic-influence}

In the early 19th century Bernard Bolzano made many innovative contributions to logic in respect to topics such as logical truth, analyticity and validity. Later in the century George Boole attempted to show that logic how logic can be expressed algebraically. The system devised by Boole, along with alterations and additions by his followers, became the Boolean algebra that is used today.

Concurrent to the development of the algebraic system of Boole, the school of Logicism emerged. One of it's proponents,  Gottlob Frege, is responsible for some of the some of the most important concepts used in logic today. Frege reasoned that expressions of the form “A is B” are not the same as “every A is B” and as such needed to be treated in a separate manner. Furthermore, “every A is B” is not an atomic proposition but rather a composite proposition which asserts that if some object satisfies the function is A then we necessarily have that this object satisfies the function is B****(). 

An implication of this distinction is the resolution of the problem of multiple generality. The problem of multiple generality is a short coming in traditional logic where, since sentences with more than one quantifier cannot be accurately represented,  there are certain intuitively valid inferences which can not be made. The notion of scope of quantifiers (Quantifiers may be applied to specific parts of a sentence as well as the whole sentence) allows such sentences to be expressed non ambiguously and the previously unachievable inferences can be made in this system.

Despite Frege attempts at a mathematically rigorous formal language, it was shown by Bertrand Russell to be inconsistent.  Two of the most important of the proposed resolutions were Zermelo set theory and another was published by Bertrand Russell and Alfred North Whitehead in the Principia Mathematica.

Zermelo set theory was the first axiomatic set theory.  It was this theory along with contributions by Abraham Fraenkel that eventually led to Zermelo-Fraenkel set theory which is now the standard form of axiomatic set theory. \cite{cantone2001set}

Bertrand Russell and Alfred North Whitehead's proposed solution attempted to resolve Russell's paradox by defining a hierarchy of types and subsequently assigning each entity a type. Aside from introducing many important notions, such as type theory, arguably it's most important contribution was the work that it inspired. Principia Mathematica, using it's more accessible notation, showed both the expressiveness of predicate logic and the deductive power of the new language while highlighting links with traditional logic. Ultimately Principia Mathematica stimulated a great deal of research within logic and related fields. \cite{sep-principia-mathematica}

Shortly after Leopold L\"{o}wenheim classified a particular subset of the logical system, now known as First-order Logic and demonstrated that for any sentence in First-order logic if it satisfiable then it is satisfiable in a countable domain. Work by Thoralf Skolem expanded on L\"{o}wenheim's research and went on to generalise results resulitng in the L\"{o}wenheim–Skolem theorem. In 1923 during his research Skolem gave the earliest first-order axiomisation of ZF set theory.\cite[p.~6]{KingShapiro95}

The L\"{o}wenheim–Skolem theorem led directly to the very important G\"{o}del completeness theorem in 1929 which showed that a given sentence of FOL is deducible in a deductive system of the language if and only if it is logically valid.  G\"{o}del, in his Incompleteness thoerems,  would go on to show that axiomatic systems of arithmetic contain a sentence which cannot be shown to be either true or false. \cite{ KingShapiro95} 

By the 1930's the modifications to Zermelo-Fraenkel and the type theoretical system meant that the two opposing systems had became similar in many aspects. The rise of the acceptance of FOL, helped by the properties such as completeness and compactness, meant it became the norm to express the Zermelo-Fraenkel set theory in FOL. While Zermelo-Fraenkel set theory can be represented in FOL, type theory is a higher order logic. Due to the similarities but simpler of nature of the former it was the FOL based Zermelo-Fraenkel set theory that became the more prevalent.\cite[p.~478]{Ferreiros01}

The work of Alfred Tarski was another influential logician whose work had wide reaching  significance, with his prestigious accounts of the concepts of truth and logical consequence perhaps being his most famous.




Aristotle's students Theophrastus and Eudemus continued Aristotle's work and made improvements to Aristotelian logic. 